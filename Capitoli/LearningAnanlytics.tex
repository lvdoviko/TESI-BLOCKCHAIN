\chapter{Learning Analytics}
La delicata tematica di privacy e sicurezza affrontata nel capitolo precedente, caratterizza fortemente il ramo
del Learning Analytics.
Questo particolare campo emergente nell'ambito dell'educazione si concentra nel raccogliere, misurare e analizzare i dati raccolti sugli studenti e sui loro corsi
di studio, con l'obiettivo di ottimizzare l'esperienza educativa e migliorare non solo i loro risultati, ma anche i metodi di insegnamento. Grazie alla crescente digitalizzazione e all'adozione di strumenti 
come i Learning Management System (LMS), i social media e i corsi online aperti e massivi (MOOCs), è possibile raccogliere una grande quantità di dati relativi al comportamento degli studenti, 
ai loro risultati e alle interazioni con i materiali didattici. 
\\Il Learning Analytics offre un enorme aiuto a tutti gli stakeholder coinvolti nel processo educativo:
\begin{itemize}
    \item Agli studenti che possono riflettere sui propri progressi e migliorare il proprio apprendimento attraverso feedback personalizzati e a loro volta dare feedback sui corsi per aiutare i docenti a migliorare i metodi di insegnamento.
    \item Ai docenti che possono adattare i contenuti dei corsi in base ai feedback e ai risultati degli studenti e identificare tempestivamente gli studenti in difficoltà.
    \item Agli amministratori accademici che possono utilizzare i dati per prendere decisioni basate sull'evidenza e sviluppare strategie efficaci per migliorare l'efficienza e la qualità dell'istruzione.
\end{itemize}
Proprio per questo le principali applicazioni del Learning Analytics sono il monitoraggio delle prestazioni individuali, la prevenzione dell'abbandono scolastico,
la personalizzazione dei percorsi educativi e l'analisi delle tecniche di valutazione e dei curricula \cite{wikipedia_learning_analytics}.

\section{Limiti del Learning Analytics}
Nonostante la disponibilità di standard di riferimento per la gestione dei dati di apprendimento su un LRS (Learning Record Store), è ancora difficile raggiungere l'interoperabilità,
ovvero la capacità di due piattaforme differenti di scambiare le informazioni in maniera indipendentemente, senza alcune limitazioni.
\\Questi problemi includono:
\begin{itemize}
    \item Collegare le storie di apprendimento di uno studente su diverse piattaforme di apprendimento in un unico percorso immutabile, in modo tale che ogni studente abbia un'unica identità nel web e non diverse identità in base alle piattaforme su cui si collega.
    \item Garantire la privacy dei registri degli studenti con facilità nel controllo degli accessi.
    \item Integrare i sistemi di ricerca e produzione per migliorare l'apprendimento.
\end{itemize}

\subsection{Collegare le storie di apprendimento }
