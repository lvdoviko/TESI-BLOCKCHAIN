% le caratteristiche richieste dall'università sono elencate qui: https://stem.elearning.unipd.it/mod/book/view.php?id=234&chapterid=46#modalita
% 12pt: font richiesto dall'università
% twoside: i margini interni ed esterni sono scambiati per le pagine "a sinistra" e "a destra"
% openright: i capitoli cominciano in pagine dispari ("a destra")
% extreport: supporta 12pt
\documentclass[12pt,a4paper,twoside,openright]{extreport}

\usepackage{changepage}
\usepackage{amsmath}                            % per avere più controllo sulle equazioni 
\usepackage{csquotes}                           % per le citazioni
\usepackage{enumitem}                           % per avere più controllo sulle enumerazioni
\usepackage[
    a4paper,
    top=2cm,bottom=2cm,
    outer=2cm,inner=3cm,
    includeheadfoot
]{geometry}                                     % margini richiesti dall'università
\usepackage{graphicx}                           % per le immagini
\usepackage{icomma}                             % per separare le cifre decimali con una virgola
\usepackage{minted}                             % per il codice con la colorazione della sintassi
\usepackage[a-1a]{pdfx}                         % formato richiesto dall'università
\usepackage[output-decimal-marker={,}]{siunitx} % per le unità di misura
\usepackage{subcaption}                         % per le sottodidascalie
\usepackage{svg}                                % per inserire file SVG

\usepackage[italian]{babel}
\selectlanguage{italian}

\usepackage[backend=bibtex,style=ieee]{biblatex}
\bibliography{bibliografia}

\usepackage{fontspec}
\setmainfont{Times New Roman} % carattere richiesto dall'università

\usepackage{setspace}
\onehalfspacing % interlinea richiesta dall'università

\sloppy % per evitare che il testo in \verb finisca oltre i margini

% questi valori vengono usati nella composizione del frontespizio
\title{Blockchain e Learning Analytics: Una Nuova Frontiera per la Gestione dei Dati}
\author{Speranza Ludovico}
\date{13/03/2025}
\newcommand{\supervisor}{Prof. Varagnolo Damiano}
%\newcommand{\assistantsupervisor}{Cognome Nome}
\usepackage{wrapfig}
\usepackage{float}

\begin{document}
    \pagenumbering{roman}
    \pagestyle{empty} % per le prime pagine, non mostrare il numero di pagina

    \input{Capitoli/frontespizio}
    \cleardoublepage
    
    \vspace*{\stretch{1}}
\begin{flushright}
    \textit{"I can accept failure, everyone fails at something. But I can't accept not trying."}\\
    --- Michael Jordan
\end{flushright}
\vspace{\stretch{1}}
    \cleardoublepage

    \pagestyle{plain} % comincia a mostrare il numero di pagina

    % NOTA: l'ambiente \abstract rimuove il numero della pagina e resetta il contatore delle pagine. 
    \chapter*{Abstract}
    La Blockchain è una tecnologia spesso definita "l’Internet del futuro", poiché rappresenta un vero e proprio stravolgimento infrastrutturale con potenziali ripercussioni in numerosi settori. La sua componente innovativa risiede nella possibilità di sviluppare applicazioni decentralizzate e sicure, che non richiedono la presenza di un intermediario. Grazie a queste caratteristiche, la Blockchain è destinata a trasformare molti aspetti della società moderna e si trova oggi al centro di discussioni tecnologiche e monetarie.
    \\Questa tesi esplora l’evoluzione della Blockchain, analizzando inizialmente l’aspetto per cui è nata, ovvero la gestione delle criptovalute, soffermandosi in particolare su Bitcoin ed Ethereum e sui loro algoritmi di validazione. Successivamente, il lavoro si concentra sul tema della protezione dei dati personali, partendo dal Web 3.0 per arrivare ad affrontare le problematiche del Learning Analytics.
    Viene fornita un’analisi critica del sito FaceItTools, proponendo una sua possibile trasformazione blockchain-oriented al fine di migliorare la gestione dei dati, garantendo maggiore trasparenza e affidabilità. In aggiunta, vengono analizzati Blockcerts, EduCTX e Sony Global Education come esempi concreti di piattaforme basate sulla blockchain.  
    \\La ricerca proposta si pone l'obiettivo di offrire una visione completa e critica delle opportunità offerte dalla Blockchain, mettendo in evidenza, in particolare, le potenzialità che essa può avere nel trattamento dei dati sensibili degli studenti in ambito scolastico.
    \cleardoublepage

    \tableofcontents
    \cleardoublepage
    
    \listoffigures
    \cleardoublepage % per assicurarsi che la numerazione araba cominci col primo capitolo
    
    \pagenumbering{arabic}
    \chapter{Introduzione}
\section{Premessa}
Nonostante il grande successo della trasformazione digitale, molti ambiti delle nostre interazioni sono rimasti datati e non si sono adattati al digitale; un esempio su tutti è quello del settore legale. Detto ciò, è importante anche notare che la digitalizzazione presenta anche lati oscuri \cite{La_quarta_rivoluzione_industriale}: i nostri dati sono controllati da organizzazioni il cui obiettivo principale è il profitto, gli hacker tentano continuamente di sottrarre informazioni personali e gli Stati monitorano le attività dei cittadini per mantenere l’ordine e il controllo.

\section{Come nasce la Blockchain}
\begin{wrapfigure}{r}{0.25\textwidth} %this figure will be at the right
    \centering
    \includegraphics[width=0.25\textwidth]{Immagini/Satoshi_Nakamoto.jpg}
    \caption{Statua raffigurante Satoshi Nakamoto}
\end{wrapfigure}
Tutto ha origine nel 2008, con la pubblicazione online dell’articolo “Bitcoin: A Peer-to-Peer Electronic Cash System” da parte di Satoshi Nakamoto, pseudonimo che potrebbe celare un individuo o un gruppo di persone. L’obiettivo del documento era quello di creare un sistema puramente peer-to-peer per il trasferimento di valore tra le parti, eliminando la necessità di intermediari come banche o istituti finanziari.
Il concetto chiave della blockchain è la possibilità di creare una generazione di piattaforme decentralizzate e disintermediate, nelle quali la fiducia tra le parti non è garantita da un’organizzazione centrale, ma dalla piattaforma stessa, attraverso un meccanismo algoritmico: essa si riferisce, in senso più ampio, all’insieme di tecnologie che rendono possibile la decentralizzazione, basandosi sul modello peer-to-peer.
In sostanza consente di immaginare un mondo organizzato in modo completamente diverso.

    \cleardoublepage
    \chapter{Blockchain}
\section{Blokchain 1.0}
La blockchain è un database completamente decentralizzato, in cui ogni nodo possiede una replica e concorda, sulla base di un algoritmo comune, un’unica versione aggiornata delle informazioni. Si tratta di una struttura dati nella quale è possibile solo aggiungere nuovi dati, senza possibilità di cancellare quelli precedenti, garantendo così uno storico completo delle modifiche.
\\Per chiarire meglio questo concetto, possiamo immaginare la blockchain come una barriera corallina: lo strato attivo corrisponde alla parte più recente, che viene continuamente modificata con nuove aggiunte, mentre gli strati più vecchi rimangono immutabili e accessibili solo per attività di consultazione.
Può essere quindi rappresentata come una lista in continua crescita di “blocchi” collegati tra loro e protetti mediante crittografia. 
\\Oltre alla decentralizzazione e all’immutabilità, il terzo grande punto di forza è il consenso: nuove transazioni possono essere registrate solo quando la maggioranza dei partecipanti alla rete dà il proprio consenso.
\\I blocchi sopra citati sono composti da 7 campi principali: 
\begin{itemize}
    \item[\textit{Block}:]  rappresenta il numero del blocco all'interno della blockchain. Ogni volta che un nuovo blocco viene aggiunto, questo numero aumenta progressivamente.
    \item[\textit{Nonce}:]  è un valore, spesso composto da 32 bit (come nel caso di Bitcoin), fondamentale perché viene cercato dai miner per generare un hash valido che soddisfi i requisiti di difficoltà della chain. Serve a modificare l’input del calcolo dell’hash e viene aggiornato continuamente durante il processo di mining.
    \item [\textit{Timestamp}:] indica il momento esatto in cui il blocco è stato aggiunto alla blockchain.
    \item [\textit{Transaction}:] contiene o un valore (ad esempio un importo monetario), o il corpo del messaggio, o uno smart contract.
    \item [\textit{Transaction n}:] si riferisce a una specifica transazione tra le diverse transazioni presenti all’interno del blocco.
    \item [\textit{Prev Hash}:] è l’hash del blocco precedente, che funge da chiave di collegamento tra i blocchi. Questo meccanismo garantisce che ogni blocco sia dipendente dal precedente, creando una catena ininterrotta.
    \item [\textit{Hash}:] rappresenta l’impronta digitale del blocco. È un valore alfanumerico unico calcolato applicando una funzione hash. L’hash è deterministico, nel senso che lo stesso input produrrà sempre lo stesso hash; tuttavia, non è possibile risalire ai dati originali partendo dall’hash. Ogni minima modifica ai dati genera un hash completamente diverso, garantendo così l’integrità e l’immutabilità delle informazioni.
\end{itemize}
\begin{figure}[t]
    \centering
    \includegraphics[width=0.8\textwidth]{Immagini/structure of block.png}
    \caption{Esempio di catena}
\end{figure}
\subsection{Bitcoin}

L’aspetto rivoluzionario di Bitcoin è la creazione di una forma monetaria completamente decentralizzata. La sua rete è composta da una serie di componenti chiamate nodi, che collaborano tra loro per raggiungere il consenso su una sequenza di transazioni.
Essendo basato sulla blockchain, Bitcoin è completamente decentralizzato: non è gestito da un’entità centrale, come avviene per esempio con le banche. 
\\Questo comporta un vantaggio significativo, ovvero una maggiore resistenza a comportamenti anomali o malintenzionati. In un sistema centralizzato, infatti, il nodo centrale rappresenta un punto di vulnerabilità che potrebbe compromettere l’intero sistema.
\\Un’altra caratteristica fondamentale di Bitcoin è la disintermediazione: ogni transazione coinvolge direttamente il mittente e il destinatario, che raggiungono il consenso tramite l’approvazione della rete. Bitcoin è quindi costituito da una rete di sistemi indipendenti, i nodi, capaci di mantenere un registro univoco e condiviso delle transazioni. Questo sistema garantisce che un singolo bitcoin non possa essere speso più di una volta.\\
In tal modo, Bitcoin soddisfa le funzioni critiche di ogni valuta:
\begin{itemize}
\item Immagazzinare valore.
\item Essere un’entità contabile.
\item Trasferire valore tra le parti.
\end{itemize}
La sua blockchain è di tipo permissionless, il che significa che chiunque può eseguire il software del nodo e collegarsi alla rete senza alcuna forma di autenticazione. In altre parole, chiunque può unirsi alla rete senza necessità di approvazioni o permessi da parte di un’autorità centrale.
\\Un elemento fondamentale è il wallet che consente agli utenti di gestire i propri bitcoin, conservati in veri e propri conti. Ogni bitcoin è associato a un address, protetto da una coppia di chiavi crittografiche.
La chiave pubblica rappresenta l’indirizzo del conto ed è accessibile a tutti, mentre la chiave privata rappresenta l’identità del proprietario ed è necessaria per accedere e autorizzare le transazioni.
\\Bitcoin utilizza la crittografia asimmetrica, un sistema inventato nel 1976 da Whitfield Diffie e Martin Hellman. Questo metodo permette di cifrare un messaggio con una chiave (pubblica) che può essere decifrato solo con l’altra chiave (privata), e viceversa, ciò determina due vantaggi principali.
Il primo è Hashing dei dati che trasforma qualsiasi quantità di dati in una stringa cifrata di dimensioni fisse, chiamata digest e comporta numerosi vantaggi in quanto è: 
\begin{itemize}
    \item[\textit{Deterministico}:] ovvero lo stesso input produce sempre lo stesso digest.
    \item [\textit{Lunghezza fissa}:] il digest ha una lunghezza predeterminata, indipendentemente dalla dimensione dell’input.
    \item [\textit{Unico}:] è estremamente improbabile che due input diversi generino lo stesso digest.
    \item [\textit{Non invertibile}:] è impossibile risalire all’input originale partendo dal digest.
    \item [\textit{Instabile}:] una minima modifica all’input genera un digest completamente diverso.
\end{itemize}
Il secondo grosso vantaggio è la firma digitale che garantisce unicità e non ripudiabilità. 
\\Quando un utente X invia a un utente Y un messaggio firmato digitalmente, X crea l’hash del messaggio e lo cifra con la propria chiave privata. Y, ricevendo il messaggio, può verificare l’hash con la chiave pubblica di X, garantendo così l’integrità del messaggio e la validità del mittente.
\begin{figure}[h]
\centering
\includegraphics[width=0.65\textwidth]{Immagini/cifratura asimmetrica.png}
\caption{Esempio di cifratura asimettrica}
\end{figure}
\\
Bitcoin è open source, il suo codice sorgente è pubblico e rilasciato sotto licenza MIT, che permette la massima libertà di utilizzo per progetti futuri.
\\Nonostante il grande successo riscosso, Bitcoin presenta diverse criticità che ne limitano l'efficacia e l'adozione su larga scala. Una delle principali problematiche riguarda la scalabilità: l'algoritmo Proof of Work (PoW) su cui si basa il sistema, pur garantendo sicurezza, tende a limitare l'efficienza complessiva, rendendo difficile gestire un alto numero di transazioni in tempi brevi.
Un altro aspetto delicato è quello relativo a privacy e visibilità. Sebbene gli account su Bitcoin siano pseudonimi, tutte le transazioni effettuate sulla rete sono pubblicamente visibili. Questo permette, tramite un'analisi approfondita dei movimenti della valuta, di risalire potenzialmente all'identità dei possessori, mettendo in discussione la reale riservatezza del sistema.
La difficoltà di aggiornamento rappresenta un'ulteriore sfida. Il complesso sistema di governance, che coinvolge minatori, sviluppatori e utenti, rende particolarmente arduo raggiungere un consenso per apportare modifiche al protocollo. Ciò può rallentare l'implementazione di miglioramenti necessari.
\\Infine, vi è il problema delle commissioni elevate, che si verifica nei periodi di intenso utilizzo della rete. In queste circostanze, effettuare transazioni rapide può diventare molto costoso, specialmente se il tasso di cambio tra Bitcoin ed Euro è particolarmente alto.

\newpage
\subsection{Proof Of Work}
Come accennato in precedenza, la sicurezza e la validità delle transazioni in Bitcoin sono garantite da un algoritmo di consenso chiamato Proof of Work (PoW) \cite{Blockchain_guida_allecosistema}. Un algoritmo di consenso è un protocollo condiviso da tutti i nodi della rete, che consente loro di concordare su una visione unica delle transazioni.
\\In primo luogo, ogni nodo condivide con gli altri nodi della rete le transazioni ricevute nell’ultimo intervallo di tempo, firmate digitalmente. Successivamente, ciascun nodo costruisce in memoria un elenco delle transazioni ricevute dagli altri nodi con cui è in comunicazione. Queste transazioni vengono ordinate e viene verificato che ogni account disponga dei fondi necessari per completare i pagamenti.
A questo punto, i nodi partecipano a una competizione crittografica per individuare, in modo casuale, il nodo che avrà il diritto di pubblicare il blocco successivo. Quando un nodo vince la competizione, individua un blocco valido e lo comunica a tutti gli altri nodi. Questi ultimi verificano il blocco e lo aggiungono alla propria copia della blockchain.
\begin{figure}[h]
\centering
\includegraphics[width=0.65\textwidth]{Immagini/proof of work scheme.jpg}
\end{figure}
\\
In questo ambito si sente spesso parlare di fork \cite{Blockchain_tecnologia_e_applicazioni_per_il_business}, ma non è sempre chiaro cosa si intenda con questo termine e cosa effettivamente comporti. Sebbene venga spesso utilizzato per indicare la divisione di una blockchain in realtà esso racchiude un insieme di diversi possibili scenari. 
\\Una fork è una situazione in cui si verifica uno dei seguenti scenari:
\begin{itemize}
\item Può accadere che due o più nodi trovino contemporaneamente una soluzione valida. In tal caso, si generano due blocchi con lo stesso genitore, creando una biforcazione nella catena, detta fork. Quando viene individuato il blocco successivo, l’algoritmo seleziona la catena più lunga (in termini di difficoltà complessiva) come valida. Questo caso viene denominato come fork regolare.
\item Le regole della Blockchain sono cambiate in maniera retrocompatibile e tutti i nodi condividono la stessa cronologia delle transazioni, in questo caso si tratta di soft fork e non c’è una divisione della blockchain.
\item Le regole della Blockchain sono cambiate in maniera non retrocompatibile, ma tutti i nodi si aggiornano alle nuove regole e condividono la stessa cronologia delle transazioni, anche in questo caso non si ha una divisione della blockchain, ma si tratta di un hard fork.
\item Le regole della blockchain sono cambiate in maniera non retrocompatibile e nodi diversi hanno opinioni diverse sulle regole della blolckchain, non condividendo la stessa cronologia delle transazioni, abbiamo quindi un hard fork con chain split, dato che avviene una divisione della chain e il consenso sulla cronologia delle transazioni e perso definitivamente.
\end{itemize}
Questo criterio, basato sul "principio di maggior lavoro", permette a tutti i nodi di concordare su una versione univoca della blockchain e di risolvere le biforcazioni. Un blocco viene considerato finalizzato quando sono stati generati almeno sei blocchi successivi. Questo livello di profondità rende economicamente insostenibile, per una minoranza di nodi malintenzionati, modificare i dati della blockchain, poiché richiederebbe un effort computazionale estremamente elevato pari al 51\% della potenza totale di calcolo del network e anche se ipoteticamente un miner riuscisse a raggiungere tale potenza non sarebbe comunque in grado di modificare le vecchie transazioni, poiché dovrebbe ricalcolare la PoW di tutti i blocchi successivi, mentre gli altri miner onesti continuano a minare sulla blockchain corretta. Un attacco di questo tipo richiederebbe l’utilizzo di una quantità incredibile di risorse per l’attacker. Se qualcuno effettivamente riuscisse a mettere insieme più del 51\% della potenza di calcolo, sarebbe molto più redditizio per lui seguire le regole della blockchain.
Nonostante i numerosi vantaggi che il Proof of Work offre, questo algoritmo di consenso presenta alcune significative criticità. La principale è il massiccio consumo di energia, un aspetto che, paradossalmente, costituisce anche uno dei suoi punti di forza, rendendo estremamente costoso e complesso attaccare la rete.
A questo si aggiunge la scarsa scalabilità del sistema, che si traduce in una certa lentezza nell’elaborazione delle transazioni e nell’aumento delle commissioni, soprattutto nei periodi di maggiore attività sulla rete. Infine, il Proof of Work tende a creare una sorta di discriminazione geografica: attualmente, la maggior parte dei miner si concentra in aree dove il costo dell’elettricità è più basso, limitando la partecipazione globale e accentuando disparità economiche e infrastrutturali \cite{Blockchain_tecnologia_e_applicazioni_per_il_business}.

\section{Blokchain 2.0}

\subsection{Ethereum e Proof Of Stake}
Bitcoin, per sua natura, non è stato concepito come un ambiente di sviluppo e offre funzionalità di programmazione molto limitate, mentre la community ha sempre voluto mantenere la piattaforma focalizzata sullo scambio di valori, lasciando spazio a progetti alternativi che miravano ad ampliare le possibilità della tecnologia blockchain.
\\È in questo scenario che nasce Ethereum, la prima blockchain progettata specificamente per supportare lo sviluppo di applicazioni decentralizzate. Co-fondata nel 2013 da Vitalik Buterin, un programmatore russo-canadese che in passato aveva collaborato con Bitcoin Magazine, Ethereum si distingue da Bitcoin per molteplici innovazioni.
\\Una delle principali novità introdotte è la possibilità di utilizzare due tipi di conti: quelli tradizionali posseduti tramite una coppia di chiavi pubblica e privata, come in Bitcoin e quelli associati agli Smart Contract \cite{Blockchain_guida_allecosistema}. 
\subsection{Smart Contracts}
Gli Smart Contract, concettualizzati per la prima volta da Nick Szabo nel 1994, sono programmi in grado di eseguire automaticamente azioni predeterminate una volta soddisfatte certe condizioni.
Questi contratti decentralizzati garantiscono che le regole siano rispettate senza la necessità di intermediari, eliminando possibili interferenze.
\\Un esempio pratico di utilizzo degli Smart Contract è il crowdfunding. Un utente potrebbe pubblicare un progetto stabilendo un obiettivo economico e un tempo limite per raggiungerlo. Lo Smart Contract, in questo caso, si occuperebbe di raccogliere i fondi dagli investitori e trattenerli fino al completamento della campagna. Se l’obiettivo viene raggiunto, i fondi vengono trasferiti automaticamente al creatore del progetto; in caso contrario, tornano ai donatori. 
\\Questa struttura elimina la necessità di intermediari e rende il processo completamente trasparente.
Ethereum utilizza Solidity, un linguaggio di programmazione simile a JavaScript, per scrivere Smart Contract, la flessibilità offerta da questa piattaforma ha rivoluzionato il mondo delle blockchain, permettendo la creazione di applicazioni decentralizzate e inaugurando l’era delle cosiddette blockchain 2.0.
\subsection{Vantaggi e svantaggi del PoS}
Inizialmente veniva utilizzato il meccanismo di consenso Proof of Work (PoW), lo stesso di Bitcoin, però per migliorare efficienza e sostenibilità, il 15 settembre 2022, con l’aggiornamento noto come The Merge, Ethereum è passato al Proof of Stake (PoS). 
Questo cambiamento ha segnato un cambio epocale per la piattaforma ed ha portato con se enormi vantaggi \cite{crypto_gateway_video}.
\\Il PoS riduce drasticamente il consumo energetico, abbassandolo del 99,9\% rispetto al PoW, ed elimina la necessità di hardware costoso per il mining, rendendo la partecipazione al consenso più accessibile.
Con l’introduzione del PoS, Ethereum ha reso possibile lo staking, un processo in cui i partecipanti bloccano una quantità di criptovaluta in uno Smart Contract per sostenere la sicurezza e il funzionamento della blockchain. Attraverso lo staking, gli utenti possono diventare validatori e partecipare attivamente alla convalida delle transazioni e alla creazione di nuovi blocchi.
\\Lo staking offre numerosi vantaggi: aumenta la sicurezza del sistema rendendo gli attacchi costosi e difficili da eseguire, favorisce la decentralizzazione permettendo a più utenti di contribuire alla rete e riduce il consumo energetico rispetto al mining tradizionale. 
I partecipanti allo staking ricevono ricompense proporzionali alla loro attività sotto forma di nuove criptovalute o ricevendo una parte delle commissioni generate dalle transazioni.
Le ricompense non si limitano ai validatori che propongono nuovi blocchi, ma includono anche coloro che verificano e confermano la validità dei blocchi proposti da altri. Questo sistema rende lo staking una delle principali motivazioni per partecipare al consenso su Ethereum, favorendo un equilibrio tra efficienza e incentivi economici.
Inoltre, il sistema garantisce una maggiore scalabilità e permette ai validatori onesti di proteggere la rete da attacchi, penalizzando economicamente i nodi malevoli o inattivi per almeno il 50\% del tempo attraverso un meccanismo chiamato slashing che risulterà totale per i primi andando a sottrarre in maniera permanente le risorse depositate nello stake, mentre sarà parziale per i secondi che si vedranno sottratti solo i guadagni ricevuti ma non saranno rimossi dalla chain.
\\Nonostante questi vantaggi, il PoS presenta alcune criticità. Essendo una tecnologia più recente rispetto al PoW è meno testata e potrebbe essere più vulnerabile a eventi imprevisti, i cosiddetti Black Swan. Inoltre, il fatto che il controllo della rete sia legato al possesso di criptovalute pone un rischio di centralizzazione: chi dispone di maggiori capitali potrebbe accumulare una quantità significativa di token, aumentando la propria influenza sul sistema \cite{kraken_docsend}.

    \cleardoublepage
    \chapter{Sicurezza}
\section{Benefici in termini di privacy e sicurezza}
L'adozione delle tecnologie basate su Distributed Ledger, come la blockchain, si è progressivamente ampliata oltre il settore delle criptovalute, trovando applicazione in ambiti come quello della protezione della privacy \cite{borroni_blockchain_2019}.
La spinta verso la decentralizzazione deriva principalmente dalle crescenti preoccupazioni degli utenti riguardo alla perdita di controllo sui propri dati personali archiviati online \cite{rodota_quattro_paradigmi_2018} \cite{alpa_identita_digitale_2017}.
A tal riguardo, la struttura stessa della blockchain offre potenzialità per tutelare la riservatezza delle informazioni personali; tuttavia, l'analisi dei metadati può renderla vulnerabile in determinati contesti. Di conseguenza, senza un'adeguata progettazione,
«decentralized infrastructures intended to promote individual privacy and autonomy might turn out to be much more vulnerable to governmental or corporate surveillance than their centralized counterparts» \cite{de_filippi_interplay_2016}.
In questa prospettiva, non va dimenticato che la natura pseudonima di molte
reti che si basano sulla blockchain consente agli individui la possibilità di
condurre le proprie transazioni su base peer-to-peer, senza la necessità di
rivelare la propria identità alle controparti.
Allo stesso tempo, per converso, la trasparenza derivante dalle distributed ledger technologies è tale che chiunque ha la possibilità di accedere alla
cronologia di tutte le transazioni memorizzate sulla blockchain, affidandosi così all’analisi dei dati in essa contenuti per ricavare informazioni potenzialmente sensibili \cite{marr_history_blockchain_2018}.
\section{Introduzione al web 3.0}
L'idea del Web 3.0 è stata inzialmente utilizzata in stretta connessione con il concetto di web semantico, terminologia creata da Berners Lee in un articolo di \textit{Scientific American} del 2001 per descrivere un nuovo web. 
La sua idea è che, così come il web 2.0 permette di collegare pagine web, a livello di visualizzazione, il web semantico deve permettere non solo di collegare pagine tra loro ma anche i dati contenuti in esse \cite{ted_youtube}.
La visone del web semantico è una visione di dati interconnessi e navigabili che possono essere usati da chiunque. L'esigenza di socializzazione dei dati è ancora più importante oggi che l'intelligenza artificiale può costruire modelli a partire dai dati grezzi, sulla base di algoritmi generali.
L'obiettivo del web 3.0, riconosciuto da Gavin Wood, co fondatore di Ethereum, è quello di creare un web decentralizzato, dove i dati sono posseduti dagli utenti e non da poche aziende. Sta nascendo la consapevolezza di re-decentralizzare i servizi web.
Questa idea è diffusa da molto tempo, ma adesso finalmente esistono le tencologie che permettono di realizzarla, come la blockchain.
Così si potrebbero ottenere numerosi vantaggi, tra i quali:
\begin{itemize}
    \item [\textit{Decentralizzazione}:] Non è necessario alcun permesso da parte di un'autorità centrale per caricare qualcosa sul web. Questo fornisce una protezione contro qualsiasi forma di censura e controllo. Il web tornerebbe ad essere un sistema neutrale.
    \item [\textit{Democratizzazione}:] E' possibile offrire un accesso a chiunque abbia una connessione a internet, senza discriminazioni su età, sesso, razza, religione e posizione geografica.
    \item [\textit{Uptime dei servizi}:] Non essendoci nodi centrali, non esiste un punto di fallimento. Se un nodo va giù, il servizio è comunque disponibile.
    \item [\textit{Possesso dei dati}:] Gli utenti riprenderebbero possesso dei propri dati potendo decidere con chi condividerli e in che modo, potendo potenzialmente guadagnare dalla vendita dei propri dati attraverso smart contracts alle grandi multinazionali come FaceBook e Google le quali hanno tantissime informazioni sugli utenti e gli advertiser che pubblicano le pubblicità sulle loro piattaforme pagano milioni per avere questi dati.
    \item [\textit{Persistenza dei dati}:] I dati non possono essere cancellati, a meno che non venga cancellata l'intera blockchain, in quanto verranno salvati in maniera ridondante su diversi nodi distribuiti indipendentemente.
\end{itemize}
\begin{figure}[h]
    \caption{Livelli del web 3.0}
    \centering
    \includegraphics[width=0.65\textwidth]{Immagini/web_30.png}
\end{figure}
L'architettura del web 3.0 non è ancora stata definita in maniera chiara e ufficiale, ma la sua caratteristica chiave sarà certamente l'assenza
di divisione netta tra utenti e fornitori di servizi. Per esempio quando ci colleghiamo a FaceBook siamo utenti e FaceBook agisce come provider 
offrendoci un servizio in cambio dei nostri dati personali. La prossima iterazione del web permetterà di eliminare questa contrapposizione netta, perchè
gli utenti potranno essere anche fornitori di servizi, almeno nelle Blockchain pubbliche.

    \cleardoublepage
    \chapter{Learning Analytics}
La delicata tematica di privacy e sicurezza affrontata nel capitolo precedente, caratterizza fortemente il ramo
del Learning Analytics.
Questo particolare campo emergente nell'ambito dell'educazione si concentra nel raccogliere, misurare e analizzare i dati raccolti sugli studenti e sui loro corsi
di studio, con l'obiettivo di ottimizzare l'esperienza educativa e migliorare non solo i loro risultati, ma anche i metodi di insegnamento. Grazie alla crescente digitalizzazione e all'adozione di strumenti 
come i Learning Management System (LMS), i social media e i corsi online aperti e massivi (MOOCs), è possibile raccogliere una grande quantità di dati relativi al comportamento degli studenti, 
ai loro risultati e alle interazioni con i materiali didattici. 
\\Il Learning Analytics offre un enorme aiuto a tutti gli stakeholder coinvolti nel processo educativo:
\begin{itemize}
    \item Agli studenti che possono riflettere sui propri progressi e migliorare il proprio apprendimento attraverso feedback personalizzati e a loro volta dare feedback sui corsi per aiutare i docenti a migliorare i metodi di insegnamento.
    \item Ai docenti che possono adattare i contenuti dei corsi in base ai feedback e ai risultati degli studenti e identificare tempestivamente gli studenti in difficoltà.
    \item Agli amministratori accademici che possono utilizzare i dati per prendere decisioni basate sull'evidenza e sviluppare strategie efficaci per migliorare l'efficienza e la qualità dell'istruzione.
\end{itemize}
Proprio per questo le principali applicazioni del Learning Analytics sono il monitoraggio delle prestazioni individuali, la prevenzione dell'abbandono scolastico,
la personalizzazione dei percorsi educativi e l'analisi delle tecniche di valutazione e dei curricula \cite{wikipedia_learning_analytics}.

\section{Limiti del Learning Analytics}
Nonostante la disponibilità di standard di riferimento per la gestione dei dati di apprendimento su un LRS (Learning Record Store), è ancora difficile raggiungere l'interoperabilità,
ovvero la capacità di due piattaforme differenti di scambiare le informazioni in maniera indipendentemente, senza alcune limitazioni.
\\Questi problemi includono:
\begin{itemize}
    \item Collegare le storie di apprendimento di uno studente su diverse piattaforme di apprendimento in un unico percorso immutabile, in modo tale che ogni studente abbia un'unica identità nel web e non diverse identità in base alle piattaforme su cui si collega.
    \item Garantire la privacy dei registri degli studenti con facilità nel controllo degli accessi.
    \item Integrare i sistemi di ricerca e produzione per migliorare l'apprendimento.
\end{itemize}

\subsection{Collegare le storie di apprendimento }

    \cleardoublepage
    \chapter{FaceItTools}
\section{Introduzione alla piattaforma}
\textit{FaceitTools.com} è una piattaforma di Larning Analytics sviluppata da studenti dell'Università di Padova che mette a disposizione dei corsi di varie materie universitarie strutturati principalmente in spidergram.
L'obiettivo è di facilitare l'apprendimento e mettere in evidenza come i corsi siano collegati tra loro e fornisce dei quiz sui quali si può essere valutati dai docenti.
\\Offre inoltre un sistema di raccolta feedback su domande d'esame archiviate in un database MongoDB che l’utente può cercare e filtrare usando una tabella Bootstrap.
\\Sulla base dei feedback viene effettuata un'analisi della difficoltà delle domande, infatti una volta inviate le risposte alle domande, viene visualizzato attraverso un grafico a barre il numero di risposte inviate e la difficoltà media percepita.
\\Questi feedback non sono utili sono agli studenti, ma anche ai docenti che possono capire se i corsi necessitano di essere rimodellati per essere più efficaci.
\begin{figure}[h]
    \centering
    \includegraphics[width=0.65\textwidth]{Immagini/FaceItTools.PNG}
    \caption{FaceItTools homepage}
\end{figure}
\subsection{Ananlisi delle problematiche}
\subsection{Benefici di una trasformazione blockchain-oriented}
\section{Raccomandazioni per gli sviluppatori}
    \cleardoublepage
    \chapter{Conclusioni}
L'esigenza di capire e di prevedere l'evoluzione futura della società è sempre più presente e pressante,
in un mondo in cui la crescita esponenziale della tecnologia, di cui Blockchain è un esempio, ci costringe ad accelerare su una strada piuttosto tortuosa e piena di insidie.
\\Con la Blockchain abbiamo l'opportunità' di digitalizzare istituzioni molto antiche, che fino ad oggi non hanno trovato un loro spazio online.
Ci ricordiamo di questo ogni volta che facciamo la fila agli uffici pubblici e ogni volta che votiamo con la matita la scheda elettorale.
\\Come abbiamo visto molte compagnie stanno già sfruttando la Blockchain per migliorare la trasparenza e la sicurezza offrendo servizi che vanno a rispondere all'esigenza del mercato di avere un controllo maggiore sui propri dati.
\\L'istruzione è uno dei tanti settori che può trarre un enorme beneficio da questa tecnologia, in quanto la Blockchain può garantire la sicurezza e la trasparenza dei dati degli studenti, permettendo loro di avere un controllo maggiore sulle proprie informazioni e di poterle condividere con chi vogliono.
\\Il medesimo concetto può essere applicato a internet, dove gli utenti non sono in grado di controllare i propri dati e non sanno come vengono utilizzati, per questo motivo il passaggio al Web 3.0 porterebbe con sè innumerevoli novità segnando sicuramente una svolta epocale per quella che è la storia di internet per come lo conosciamo oggi.
\\Il vento di novità che soffia sopra questi argomenti è decisamente rinfrescante e pieno di senso di sfida che spinge a capire meglio l'impatto del cambiamento e come questo possa essere previsto e indirizzato.
\\Questa tesi cerca di racchiudere la maggior parte dei concetti più importanti, focalizzandosi sulla protezione della privacy dei dati personali e sulla trasparenza delle informazioni,
ma non può essere considerata una spiegazione esaustiva, in quanto l'argomento è estremamente vario e complesso. 
\\L'obiettivo è dunque quello di fornire uno spunto di partenza per altri studenti e ricercatori che vogliono approfondire l'argomento, proponendo delle implementazioni pratiche alle soluzioni proposte riguardo i problemi affrontati.
    \cleardoublepage
    \printbibliography[heading=bibintoc]
    \cleardoublepage
    \chapter*{Ringraziamenti}
    \addcontentsline{toc}{chapter}{Ringraziamenti}
    Eccomi giunto alla fine di questa tesi e di questi splendidi 3 anni (e mezzo) di Università. Vorrei dedicare quest'ultima pagina per ringraziare tutte le persone che hanno dato il loro contributo, consciamente e inconsciamente, al raggiungimento di questo splendido traguardo.
    \\Ci tenevo inanzitutto a ringraziare di cuore il Prof. Damiano per avermi seguito e supportato in maniera impeccabile durante tutta la stesura di questa tesi, per avermi dato la possibilità di esplorare questo argomento che avevo scelto in maniera individuale e per avermi sempre spronato a dare il meglio di me. 
    \\Ringrazio poi tutta la mia famiglia per non avermi mai fatto mancare niente, il supporto, l'affetto e soprattutto le risorse economiche per poter studiare e laurearmi; in particolare mio fratello Ermanno per avermi sempre dato consigli preziosi tra cui lo spunto per scegliere questo argomento di tesi.
    \\Un mio ringraziamento speciale va ad Arianna per tutti i passaggi offerti, per tutti i caricatori prestati e per avermi sopportato in tutti i momenti di stress.
    \\Un sincero grazie ai miei carissimi compagni di corso, ma in particolare a M1, M2, Davide e Simiz, perchè senza di loro e senza le infinite ore di studio passate insieme probabilmente starei ancora preparando l'esame di Architettura degli Elaboratori.
    \\E poi volevo ringraziare tutti voi, partendo dal Misetto che nonostante la distanza è sempre stato vicino incoraggiandomi durante le nostre quotidiane chiamate serali, per poi arrivare a Blak, Manuel, Mattia, Noel, Sante, Sir, Ale, Ana, Angy, Anna, Carli e Greta, per tutti i venerdi e sabati sera passati insieme tra le mille risate, perchè come disse il grande Alex, un atleta triste è un atleta che parte sconfitto.
    \\Avete dato un contributo pazzesco e vi garantisco che se il voto fosse stato assegnato in base al valore del team che mi ha sostenuto, oggi sarei uscito con la Lode. Peccato abbiano valutato solo me.    \cleardoublepage
\end{document}
