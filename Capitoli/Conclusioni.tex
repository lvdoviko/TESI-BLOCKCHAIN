\chapter{Conclusioni}
L'esigenza di capire e di prevedere l'evoluzione futura della società è sempre più presente e pressante,
in un mondo in cui la crescita esponenziale della tecnologia, di cui Blockchain è un esempio, ci costringe ad accelerare su una strada piuttosto tortuosa e piena di insidie.
\\Con la Blockchain abbiamo l'opportunità di digitalizzare istituzioni molto antiche, che fino ad oggi non hanno trovato un loro spazio online.
Ci ricordiamo di questo ogni volta che facciamo la fila agli uffici pubblici e ogni volta che votiamo con la matita sulla scheda elettorale.
\\Uno degli aspetti più significativi che ho appreso scrivendo questa tesi è che la blockchain non è una soluzione universale a tutti i problemi, ma piuttosto un nuovo paradigma che deve essere adottato con consapevolezza. 
Se da un lato offre strumenti potentissimi per decentralizzare il controllo e restituire agli utenti la proprietà dei propri dati, dall’altro solleva questioni di scalabilità, governance e sostenibilità energetica che devono essere affrontate con soluzioni mirate.
\\Come abbiamo visto, molte compagnie stanno già sfruttando la Blockchain per migliorare la trasparenza e la sicurezza, offrendo servizi che vanno a rispondere all'esigenza del mercato di avere un controllo maggiore sui propri dati.
\\L'istruzione è uno dei tanti settori che può trarre un enorme beneficio da questa tecnologia, in quanto la Blockchain può garantire la sicurezza e la trasparenza dei dati degli studenti, permettendo loro di avere un controllo maggiore sulle proprie informazioni e di poterle condividere con chi vogliono.
Sistemi come EduCTX, Sony Global Education e Blockcerts dimostrano che è possibile costruire un ecosistema più trasparente ed efficiente, in cui gli studenti possano avere pieno controllo sulle proprie informazioni e i certificati possano essere verificati in modo istantaneo e sicuro. 
\\La trasformazione blockchain-oriented della piattaforma FaceItTools, analizzata in questa tesi, di cui ho fornito una traccia da seguire prendendo spunto dai sistemi già presenti sul mercato è un esempio concreto di come questa tecnologia possa migliorare l’affidabilità e la privacy nel settore educativo.
\\Un'altra tematica che ho approfondito è il delicato equilibrio tra privacy e trasparenza. 
Se da un lato la blockchain offre una protezione dei dati più solida rispetto ai sistemi centralizzati, dall’altro è essenziale garantire che i meccanismi di accesso e autorizzazione siano adeguatamente progettati per evitare abusi e violazioni della privacy.
\\Proprio a tal proposito ho affrontato l'attuale stato di internet, dove gli utenti non sono in grado di controllare i propri dati e non sanno come vengono utilizzati, per questo motivo il passaggio al Web 3.0 porterebbe con sè innumerevoli novità segnando sicuramente una svolta epocale per la storia di internet per come lo conosciamo oggi.
\\Il vento di novità che soffia sopra questi argomenti è decisamente rinfrescante e pieno di senso di sfida che spinge a capire meglio l'impatto del cambiamento e come questo possa essere previsto e indirizzato.
\\Questa tesi cerca di racchiudere la maggior parte dei concetti più importanti, focalizzandosi sulla protezione della privacy dei dati personali e sulla trasparenza delle informazioni,
ma non può essere considerata una spiegazione esaustiva, in quanto l'argomento è estremamente vario e complesso. 
\\L'obiettivo è dunque quello di fornire uno spunto di partenza per altri studenti e ricercatori che vogliono approfondire l'argomento, proponendo delle implementazioni pratiche alle soluzioni proposte riguardo i problemi affrontati e a una possibile trasformazione blockchain-oriented della piattaforma FaceItTools, ma ricordando che l’innovazione tecnologica, per quanto avanzata, deve sempre essere accompagnata da un’etica solida e da una consapevolezza delle sue implicazioni sociali. 
La blockchain ha il potenziale per democratizzare l’accesso ai servizi digitali e ridurre le asimmetrie di potere, ma solo se utilizzata con responsabilità e con una visione che metta al centro gli utenti e la loro libertà.