% le caratteristiche richieste dall'università sono elencate qui: https://stem.elearning.unipd.it/mod/book/view.php?id=234&chapterid=46#modalita
% 12pt: font richiesto dall'università
% twoside: i margini interni ed esterni sono scambiati per le pagine "a sinistra" e "a destra"
% openright: i capitoli cominciano in pagine dispari ("a destra")
% extreport: supporta 12pt
\documentclass[12pt,a4paper,twoside,openright]{extreport}

\usepackage{amsmath}                            % per avere più controllo sulle equazioni 
\usepackage{csquotes}                           % per le citazioni
\usepackage{enumitem}                           % per avere più controllo sulle enumerazioni
\usepackage[
    a4paper,
    top=2cm,bottom=2cm,
    outer=2cm,inner=3cm,
    includeheadfoot
]{geometry}                                     % margini richiesti dall'università
\usepackage{graphicx}                           % per le immagini
\usepackage{icomma}                             % per separare le cifre decimali con una virgola
\usepackage{minted}                             % per il codice con la colorazione della sintassi
\usepackage[a-1a]{pdfx}                         % formato richiesto dall'università
\usepackage[output-decimal-marker={,}]{siunitx} % per le unità di misura
\usepackage{subcaption}                         % per le sottodidascalie
\usepackage{svg}                                % per inserire file SVG

\usepackage[italian]{babel}
\selectlanguage{italian}

\usepackage[backend=bibtex,style=ieee]{biblatex}
\bibliography{bibliografia}

\usepackage{fontspec}
\setmainfont{Times New Roman} % carattere richiesto dall'università

\usepackage{setspace}
\onehalfspacing % interlinea richiesta dall'università

\sloppy % per evitare che il testo in \verb finisca oltre i margini

% questi valori vengono usati nella composizione del frontespizio
\title{Blockchain e Learning Analytics: Una Nuova Frontiera per la Gestione dei Dati}
\author{Speranza Ludovico}
\date{GG/MM/AAAA}
\newcommand{\supervisor}{Prof. Varagnolo Damiano}
\newcommand{\assistantsupervisor}{Cognome Nome}
\usepackage{wrapfig}
\usepackage{float}

\begin{document}
    \pagenumbering{roman}
    \pagestyle{empty} % per le prime pagine, non mostrare il numero di pagina

    \input{Capitoli/frontespizio}
    \cleardoublepage
    
    \vspace*{\stretch{1}}
\begin{flushright}
    \textit{"I can accept failure, everyone fails at something. But I can't accept not trying."}\\
    --- Michael Jordan
\end{flushright}
\vspace{\stretch{1}}
    \cleardoublepage

    \pagestyle{plain} % comincia a mostrare il numero di pagina

    % NOTA: l'ambiente \abstract rimuove il numero della pagina e resetta il contatore delle pagine. 
    \chapter*{Abstract}
    La Blockchain è una tecnologia spesso definita "l’Internet del futuro", poiché rappresenta un vero e proprio stravolgimento infrastrutturale con potenziali ripercussioni in numerosi settori. La sua componente innovativa risiede nella possibilità di sviluppare applicazioni decentralizzate e sicure, che non richiedono la presenza di un intermediario. Grazie a queste caratteristiche, la Blockchain è destinata a trasformare molti aspetti della società moderna e si trova oggi al centro di discussioni tecnologiche e monetarie.
    \\Questa tesi esplora l’evoluzione della Blockchain, analizzando inizialmente l’aspetto per cui è nata, ovvero la gestione delle criptovalute, soffermandosi in particolare su Bitcoin ed Ethereum e sui loro algoritmi di validazione. Successivamente, il lavoro si concentra sul tema della protezione dei dati personali, partendo dal Web 3.0 per arrivare ad affrontare le problematiche del Learning Analytics.
    Viene fornita un’analisi critica del sito FaceItTools, proponendo una sua possibile trasformazione blockchain-oriented al fine di migliorare la gestione dei dati, garantendo maggiore trasparenza e affidabilità. In aggiunta, vengono analizzati Blockcerts, EduCTX e Sony Global Education come esempi concreti di piattaforme basate sulla blockchain.  
    \\La ricerca proposta si pone l'obiettivo di offrire una visione completa e critica delle opportunità offerte dalla Blockchain, mettendo in evidenza, in particolare, le potenzialità che essa può avere nel trattamento dei dati sensibili degli studenti in ambito scolastico.
    \cleardoublepage

    \tableofcontents
    \cleardoublepage
    
    \listoffigures
    \cleardoublepage % per assicurarsi che la numerazione araba cominci col primo capitolo
    
    \pagenumbering{arabic}

    \input{Capitoli/Introduzione_Blockchain}
    \chapter{Sicurezza}
\section{Benefici in termini di privacy e sicurezza}
L'adozione delle tecnologie basate su Distributed Ledger, come la blockchain, si è progressivamente ampliata oltre il settore delle criptovalute, trovando applicazione in ambiti come quello della protezione della privacy \cite{borroni_blockchain_2019}.
La spinta verso la decentralizzazione deriva principalmente dalle crescenti preoccupazioni degli utenti riguardo alla perdita di controllo sui propri dati personali archiviati online \cite{rodota_quattro_paradigmi_2018} \cite{alpa_identita_digitale_2017}.
A tal riguardo, la struttura stessa della blockchain offre potenzialità per tutelare la riservatezza delle informazioni personali; tuttavia, l'analisi dei metadati può renderla vulnerabile in determinati contesti. Di conseguenza, senza un'adeguata progettazione,
«decentralized infrastructures intended to promote individual privacy and autonomy might turn out to be much more vulnerable to governmental or corporate surveillance than their centralized counterparts» \cite{de_filippi_interplay_2016}.
In questa prospettiva, non va dimenticato che la natura pseudonima di molte
reti che si basano sulla blockchain consente agli individui la possibilità di
condurre le proprie transazioni su base peer-to-peer, senza la necessità di
rivelare la propria identità alle controparti.
Allo stesso tempo, per converso, la trasparenza derivante dalle distributed ledger technologies è tale che chiunque ha la possibilità di accedere alla
cronologia di tutte le transazioni memorizzate sulla blockchain, affidandosi così all’analisi dei dati in essa contenuti per ricavare informazioni potenzialmente sensibili \cite{marr_history_blockchain_2018}.
\section{Introduzione al web 3.0}
L'idea del Web 3.0 è stata inzialmente utilizzata in stretta connessione con il concetto di web semantico, terminologia creata da Berners Lee in un articolo di \textit{Scientific American} del 2001 per descrivere un nuovo web. 
La sua idea è che, così come il web 2.0 permette di collegare pagine web, a livello di visualizzazione, il web semantico deve permettere non solo di collegare pagine tra loro ma anche i dati contenuti in esse \cite{ted_youtube}.
La visone del web semantico è una visione di dati interconnessi e navigabili che possono essere usati da chiunque. L'esigenza di socializzazione dei dati è ancora più importante oggi che l'intelligenza artificiale può costruire modelli a partire dai dati grezzi, sulla base di algoritmi generali.
L'obiettivo del web 3.0, riconosciuto da Gavin Wood, co fondatore di Ethereum, è quello di creare un web decentralizzato, dove i dati sono posseduti dagli utenti e non da poche aziende. Sta nascendo la consapevolezza di re-decentralizzare i servizi web.
Questa idea è diffusa da molto tempo, ma adesso finalmente esistono le tencologie che permettono di realizzarla, come la blockchain.
Così si potrebbero ottenere numerosi vantaggi, tra i quali:
\begin{itemize}
    \item [\textit{Decentralizzazione}:] Non è necessario alcun permesso da parte di un'autorità centrale per caricare qualcosa sul web. Questo fornisce una protezione contro qualsiasi forma di censura e controllo. Il web tornerebbe ad essere un sistema neutrale.
    \item [\textit{Democratizzazione}:] E' possibile offrire un accesso a chiunque abbia una connessione a internet, senza discriminazioni su età, sesso, razza, religione e posizione geografica.
    \item [\textit{Uptime dei servizi}:] Non essendoci nodi centrali, non esiste un punto di fallimento. Se un nodo va giù, il servizio è comunque disponibile.
    \item [\textit{Possesso dei dati}:] Gli utenti riprenderebbero possesso dei propri dati potendo decidere con chi condividerli e in che modo, potendo potenzialmente guadagnare dalla vendita dei propri dati attraverso smart contracts alle grandi multinazionali come FaceBook e Google le quali hanno tantissime informazioni sugli utenti e gli advertiser che pubblicano le pubblicità sulle loro piattaforme pagano milioni per avere questi dati.
    \item [\textit{Persistenza dei dati}:] I dati non possono essere cancellati, a meno che non venga cancellata l'intera blockchain, in quanto verranno salvati in maniera ridondante su diversi nodi distribuiti indipendentemente.
\end{itemize}
\begin{figure}[h]
    \caption{Livelli del web 3.0}
    \centering
    \includegraphics[width=0.65\textwidth]{Immagini/web_30.png}
\end{figure}
L'architettura del web 3.0 non è ancora stata definita in maniera chiara e ufficiale, ma la sua caratteristica chiave sarà certamente l'assenza
di divisione netta tra utenti e fornitori di servizi. Per esempio quando ci colleghiamo a FaceBook siamo utenti e FaceBook agisce come provider 
offrendoci un servizio in cambio dei nostri dati personali. La prossima iterazione del web permetterà di eliminare questa contrapposizione netta, perchè
gli utenti potranno essere anche fornitori di servizi, almeno nelle Blockchain pubbliche.

    \chapter{Learning Analytics}
La delicata tematica di privacy e sicurezza affrontata nel capitolo precedente, caratterizza fortemente il ramo
del Learning Analytics.
Questo particolare campo emergente nell'ambito dell'educazione si concentra nel raccogliere, misurare e analizzare i dati raccolti sugli studenti e sui loro corsi
di studio, con l'obiettivo di ottimizzare l'esperienza educativa e migliorare non solo i loro risultati, ma anche i metodi di insegnamento. Grazie alla crescente digitalizzazione e all'adozione di strumenti 
come i Learning Management System (LMS), i social media e i corsi online aperti e massivi (MOOCs), è possibile raccogliere una grande quantità di dati relativi al comportamento degli studenti, 
ai loro risultati e alle interazioni con i materiali didattici. 
\\Il Learning Analytics offre un enorme aiuto a tutti gli stakeholder coinvolti nel processo educativo:
\begin{itemize}
    \item Agli studenti che possono riflettere sui propri progressi e migliorare il proprio apprendimento attraverso feedback personalizzati e a loro volta dare feedback sui corsi per aiutare i docenti a migliorare i metodi di insegnamento.
    \item Ai docenti che possono adattare i contenuti dei corsi in base ai feedback e ai risultati degli studenti e identificare tempestivamente gli studenti in difficoltà.
    \item Agli amministratori accademici che possono utilizzare i dati per prendere decisioni basate sull'evidenza e sviluppare strategie efficaci per migliorare l'efficienza e la qualità dell'istruzione.
\end{itemize}
Proprio per questo le principali applicazioni del Learning Analytics sono il monitoraggio delle prestazioni individuali, la prevenzione dell'abbandono scolastico,
la personalizzazione dei percorsi educativi e l'analisi delle tecniche di valutazione e dei curricula \cite{wikipedia_learning_analytics}.

\section{Limiti del Learning Analytics}
Nonostante la disponibilità di standard di riferimento per la gestione dei dati di apprendimento su un LRS (Learning Record Store), è ancora difficile raggiungere l'interoperabilità,
ovvero la capacità di due piattaforme differenti di scambiare le informazioni in maniera indipendentemente, senza alcune limitazioni.
\\Questi problemi includono:
\begin{itemize}
    \item Collegare le storie di apprendimento di uno studente su diverse piattaforme di apprendimento in un unico percorso immutabile, in modo tale che ogni studente abbia un'unica identità nel web e non diverse identità in base alle piattaforme su cui si collega.
    \item Garantire la privacy dei registri degli studenti con facilità nel controllo degli accessi.
    \item Integrare i sistemi di ricerca e produzione per migliorare l'apprendimento.
\end{itemize}

\subsection{Collegare le storie di apprendimento }

    \chapter{FaceItTools}
\section{Introduzione alla piattaforma}
\textit{FaceitTools.com} è una piattaforma di Larning Analytics sviluppata da studenti dell'Università di Padova che mette a disposizione dei corsi di varie materie universitarie strutturati principalmente in spidergram.
L'obiettivo è di facilitare l'apprendimento e mettere in evidenza come i corsi siano collegati tra loro e fornisce dei quiz sui quali si può essere valutati dai docenti.
\\Offre inoltre un sistema di raccolta feedback su domande d'esame archiviate in un database MongoDB che l’utente può cercare e filtrare usando una tabella Bootstrap.
\\Sulla base dei feedback viene effettuata un'analisi della difficoltà delle domande, infatti una volta inviate le risposte alle domande, viene visualizzato attraverso un grafico a barre il numero di risposte inviate e la difficoltà media percepita.
\\Questi feedback non sono utili sono agli studenti, ma anche ai docenti che possono capire se i corsi necessitano di essere rimodellati per essere più efficaci.
\begin{figure}[h]
    \centering
    \includegraphics[width=0.65\textwidth]{Immagini/FaceItTools.PNG}
    \caption{FaceItTools homepage}
\end{figure}
\subsection{Ananlisi delle problematiche}
\subsection{Benefici di una trasformazione blockchain-oriented}
\section{Raccomandazioni per gli sviluppatori}
    \cleardoublepage
    \chapter{Conclusioni}
L'esigenza di capire e di prevedere l'evoluzione futura della società è sempre più presente e pressante,
in un mondo in cui la crescita esponenziale della tecnologia, di cui Blockchain è un esempio, ci costringe ad accelerare su una strada piuttosto tortuosa e piena di insidie.
\\Con la Blockchain abbiamo l'opportunità' di digitalizzare istituzioni molto antiche, che fino ad oggi non hanno trovato un loro spazio online.
Ci ricordiamo di questo ogni volta che facciamo la fila agli uffici pubblici e ogni volta che votiamo con la matita la scheda elettorale.
\\Come abbiamo visto molte compagnie stanno già sfruttando la Blockchain per migliorare la trasparenza e la sicurezza offrendo servizi che vanno a rispondere all'esigenza del mercato di avere un controllo maggiore sui propri dati.
\\L'istruzione è uno dei tanti settori che può trarre un enorme beneficio da questa tecnologia, in quanto la Blockchain può garantire la sicurezza e la trasparenza dei dati degli studenti, permettendo loro di avere un controllo maggiore sulle proprie informazioni e di poterle condividere con chi vogliono.
\\Il medesimo concetto può essere applicato a internet, dove gli utenti non sono in grado di controllare i propri dati e non sanno come vengono utilizzati, per questo motivo il passaggio al Web 3.0 porterebbe con sè innumerevoli novità segnando sicuramente una svolta epocale per quella che è la storia di internet per come lo conosciamo oggi.
\\Il vento di novità che soffia sopra questi argomenti è decisamente rinfrescante e pieno di senso di sfida che spinge a capire meglio l'impatto del cambiamento e come questo possa essere previsto e indirizzato.
\\Questa tesi cerca di racchiudere la maggior parte dei concetti più importanti, focalizzandosi sulla protezione della privacy dei dati personali e sulla trasparenza delle informazioni,
ma non può essere considerata una spiegazione esaustiva, in quanto l'argomento è estremamente vario e complesso. 
\\L'obiettivo è dunque quello di fornire uno spunto di partenza per altri studenti e ricercatori che vogliono approfondire l'argomento, proponendo delle implementazioni pratiche alle soluzioni proposte riguardo i problemi affrontati.
    \cleardoublepage
    \printbibliography[heading=bibintoc]
\end{document}
